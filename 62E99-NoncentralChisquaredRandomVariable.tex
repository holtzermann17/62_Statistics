\documentclass[12pt]{article}
\usepackage{pmmeta}
\pmcanonicalname{NoncentralChisquaredRandomVariable}
\pmcreated{2013-03-22 14:56:16}
\pmmodified{2013-03-22 14:56:16}
\pmowner{CWoo}{3771}
\pmmodifier{CWoo}{3771}
\pmtitle{non-central chi-squared random variable}
\pmrecord{11}{36628}
\pmprivacy{1}
\pmauthor{CWoo}{3771}
\pmtype{Definition}
\pmcomment{trigger rebuild}
\pmclassification{msc}{62E99}
\pmclassification{msc}{60E05}
\pmsynonym{non-central chi-squared distribution}{NoncentralChisquaredRandomVariable}
\pmdefines{non-centrality parameter}

% this is the default PlanetMath preamble.  as your knowledge
% of TeX increases, you will probably want to edit this, but
% it should be fine as is for beginners.

% almost certainly you want these
\usepackage{amssymb,amscd}
\usepackage{amsmath}
\usepackage{amsfonts}

% used for TeXing text within eps files
%\usepackage{psfrag}
% need this for including graphics (\includegraphics)
\usepackage{graphicx}
\usepackage{float}
% for neatly defining theorems and propositions
%\usepackage{amsthm}
% making logically defined graphics
%%%\usepackage{xypic}

% there are many more packages, add them here as you need them

% define commands here
\begin{document}
\PMlinkescapeword{number}
\PMlinkescapeword{real part}
\PMlinkescapeword{integers}
\PMlinkescapeword{variable}
\PMlinkescapeword{degree}
\PMlinkescapeword{normals}
\PMlinkescapeword{density}
\PMlinkescapeword{restricted}

Let $X_1,\ldots,X_n$ be IID random variables, each with the standard normal distribution.  Then, for any $\boldsymbol{\mu}\in\mathbb{R}^n$, the random variable $X$ defined by 
\begin{equation*}
X=\sum_{i=1}^{n}(X_i+\mu_i)^2
\end{equation*}
is called a \emph{non-central chi-squared random variable}.
Its distribution depends only on the number of degrees of freedom $n$ and non-centrality parameter $\lambda\equiv\Vert\boldsymbol{\mu}\Vert$. This is denoted by $\chi^2(n,\lambda)$ and has moment generating function
\begin{equation}\label{eq:1}
\operatorname{M}_X(t)\equiv\mathbb{E}\left[e^{tX}\right]=\left(1-2t\right)^{-\frac{n}{2}}\exp\left(\frac{\lambda t}{1-2t}\right),
\end{equation}
which is defined for all $t\in\mathbb{C}$ with real part less than $1/2$.
More generally, for any $n,\lambda\ge 0$, not necessarily integers, a random variable has the  \emph{non-central chi-squared distribution}, $\chi^2(n,\lambda)$, if its moment generating function is given by (\ref{eq:1}).

A non-central chi-squared random variable for any $n,\lambda\ge 0$ can be constructed as follows. Let $Y$ be a (central) chi-squared variable with degree $n$, $Z_1,Z_2,\ldots$ be standard normals, and $N$ have the $\textrm{Poisson}(\lambda/2)$ distribution. If these are all independent then
\begin{equation*}
X\equiv Y+\sum_{k=1}^{2N}Z_k^2.
\end{equation*}
has the $\chi^2(n,\lambda)$ distribution. Correspondingly, the probability density function for $X$ is
\begin{equation}\label{eq:2}
f_X(x)=\sum_{k=0}^\infty \frac{\lambda^k}{2^k k!}e^{-\lambda/2} f_{n+2k}(x),
\end{equation}
where $x>0$ and $f_k$ is the probability density of the $\chi^2_{(k)}$ distribution.
Alternatively, this can be expressed as
\begin{equation*}
f_X(x)=\frac{1}{2}e^{-(x+\lambda)/2}(x/\lambda)^{n/4-1/2}I_{n/2-1}\left(\sqrt{\lambda x}\right).
\end{equation*}
where $I_\nu$ is a modified Bessel function of the first kind,
\begin{equation*}
I_\nu(x)=\sum_{k=0}^\infty\frac{\left(x/2\right)^{\nu+2k}}{k!\,\Gamma\left(\nu+k+1\right)}.
\end{equation*}

\begin{figure}[H]
\centering
\includegraphics{ncchisquared}
\caption{Densities of the non-central chi-squared distribution $\chi^2(n,\lambda)$.}
\end{figure}

\textbf{Remarks}
\begin{enumerate}
\item $\chi^2(n,\lambda)$ has mean $n+\lambda$ and variance $2n+4\lambda$.
\item $\chi^2(n,0)=\chi^2_{(n)}$.  The (central) chi-squared random variable is a special case of the non-central chi-squared random variable, when the non-centrality parameter $\lambda=0$.
\item (The reproductive property of chi-squared distributions).  If $Z_1,\ldots,Z_m$ are non-central chi-squared random variables such that each $Z_i\sim\chi^2(n_i,\lambda_i)$, then their total $Z=\sum Z_i$ is also a non-central chi-squared random variable with distribution $\chi^2(\sum n_i, \sum \lambda_i)$.
\item If $n>0$ then the $\chi^2(n,\lambda)$ distribution is restricted to the domain $(0,\infty)$ with probability density function (\ref{eq:2}). On the other hand, if $n=0$, then there is also an atom at $0$,
\begin{equation*}
\mathbb{P}(X=0)=\lim_{t\rightarrow-\infty}\operatorname{M}_X(t)=e^{-\lambda/2}.
\end{equation*}
\item If $\boldsymbol{x}$ is a multivariate normally distributed $n$-dimensional random vector with distribution $\boldsymbol{N(\mu,V)}$ where $\boldsymbol{\mu}$ is the mean vector and $\boldsymbol{V}$ is the $n\times n$ covariance matrix.  Suppose that $\boldsymbol{V}$ is singular, with $k$ = rank of $V<n$.  Then $\boldsymbol{x^{\operatorname{T}}V^{-}x}$ is a non-central chi-squared random variable, where $\boldsymbol{V^{-}}$ is a generalized inverse of $\boldsymbol{V}$.  Its distribution has $k$ degrees of freedom with non-centrality parameter $\lambda=\boldsymbol{\mu^{\operatorname{T}}V^{-}\mu}$.
\end{enumerate}

%%%%%
%%%%%
\end{document}
