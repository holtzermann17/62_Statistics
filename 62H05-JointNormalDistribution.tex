\documentclass[12pt]{article}
\usepackage{pmmeta}
\pmcanonicalname{JointNormalDistribution}
\pmcreated{2013-03-22 15:22:34}
\pmmodified{2013-03-22 15:22:34}
\pmowner{gel}{22282}
\pmmodifier{gel}{22282}
\pmtitle{joint normal distribution}
\pmrecord{14}{37204}
\pmprivacy{1}
\pmauthor{gel}{22282}
\pmtype{Definition}
\pmcomment{trigger rebuild}
\pmclassification{msc}{62H05}
\pmclassification{msc}{60E05}
\pmsynonym{multivariate Gaussian distribution}{JointNormalDistribution}
\pmrelated{NormalRandomVariable}
\pmdefines{jointly normal}
\pmdefines{multivariate normal distribution}

\endmetadata

\usepackage{amssymb,amscd}
\usepackage{amsmath}
\usepackage{amsfonts}

% used for TeXing text within eps files
%\usepackage{psfrag}
% need this for including graphics (\includegraphics)
\usepackage{graphicx}
\usepackage{float}
% for neatly defining theorems and propositions
%\usepackage{amsthm}
% making logically defined graphics
%%%\usepackage{xypic}

% define commands here

\newcommand{\trnsp}[1]{#1^{\operatorname{T}}}
\newcommand{\bs}[1]{\boldsymbol{#1}}

\begin{document}
\PMlinkescapeword{normal}
A finite set of random variables $X_1,\ldots,X_n$ are said to have a
\emph{joint normal distribution} or \emph{multivariate normal
distribution} if all real linear combinations
\begin{equation*}
\lambda_1X_1 + \lambda_2X_2 +\cdots+\lambda_nX_n
\end{equation*}
are \PMlinkname{normal}{NormalRandomVariable}. This implies, in particular, that the individual random variables $X_i$ are each normally distributed. However, the converse is not not true and sets of normally distributed random variables need not, in general, be jointly normal.

If $\bs{X}=(X_1,X_2,\ldots,X_n)$ is joint normal, then its probability distribution is uniquely determined by the means $\bs{\mu}\in\mathbb{R}^n$ and the $n\times n$ positive semidefinite covariance matrix $\bs{\Sigma}$,
\begin{align*}
&\mu_i=\mathbb{E}[X_i],\\
&\Sigma_{ij}=\operatorname{Cov}(X_i,X_j)=\mathbb{E}[X_iX_j]-\mathbb{E}[X_i]\mathbb{E}[X_j].
\end{align*}
Then, the joint normal distribution is commonly denoted as $\operatorname{N}(\bs{\mu},\bs{\Sigma})$. Conversely, this distribution exists for any such $\bs{\mu}$ and $\bs{\Sigma}$.

\begin{figure}[H]
\centering
\includegraphics[bb = 50 568 320 770,clip=,scale=1.2]{jointnormal}
\caption{Density of joint normal variables $X,Y$ with $\operatorname{Var}(X)=2$, $\operatorname{Var}(Y)=1$ and $\operatorname{Cov}(X,Y)=-1$.}
\end{figure}

The joint normal distribution has the following properties:
\begin{enumerate}
\item If $\bs{X}$ has the $\operatorname{N}(\bs{\mu},\bs{\Sigma})$ distribution for nonsigular $\bs{\Sigma}$ then it has the multidimensional Gaussian probability density function
\begin{equation*}
f_{\bs{X}}(\bs{x})=\frac{1
}{\sqrt{(2\pi)^n \det{\bs{(\Sigma})}}}
\exp\left(-\frac{1}{2}\trnsp{(\bs{x} -  \bs{\mu})}\bs{\Sigma}^{-1} (\bs{x} -  \bs{\mu})\right).
\end{equation*}
\item If $\bs{X}$ has the $\operatorname{N}(\bs{\mu},\bs{\Sigma})$ distribution and $\bs{\lambda}\in\mathbb{R}^n$ then
\begin{equation*}
\bs{\lambda}\cdot\bs{X}=\lambda_1X_1+\cdots+\lambda_nX_n\sim\operatorname{N}(\bs{\lambda}\cdot\bs{\mu},\trnsp{\bs{\lambda}}\bs{\Sigma}\bs{\lambda}).
\end{equation*}
\item Sets of linear combinations of joint normals are themselves joint normal. In particular, if $\bs{X}\sim\operatorname{N}(\bs{\mu},\bs{\Sigma})$ and $A$ is an $m\times n$ matrix, then $A\bs{X}$ has the joint normal distribution $\operatorname{N}(A\bs{\mu},A\bs{\Sigma}\trnsp{A})$.
\item The characteristic function is given by
\begin{equation*}
\varphi_{\bs{X}}(\bs{a})\equiv\mathbb{E}\left[\exp(i\bs{a}\cdot\bs{X})\right]=\exp\left(i\bs{a}\cdot\bs{\mu}-\frac{1}{2}\trnsp{\bs{a}}\bs{\Sigma}\bs{a}\right),
\end{equation*}
for $\bs{X}\sim\operatorname{N}(\bs{\mu},\bs{\Sigma})$ and any $\bs{a}\in\mathbb{C}^n$.
\item A pair $X,Y$ of jointly normal random variables are independent if and only if they have zero covariance.
\item Let $\bs{X}$ be a random vector whose distribution is jointly
normal.  Suppose the coordinates of $\bs{X}$ are partitioned
into two groups, forming random vectors $\bs{X_1}$ and
$\bs{X_2}$, then the conditional distribution of
$\bs{X_1}$ given $\bs{X_2}=\bs{c}$ is
jointly normal.
\end{enumerate}

%%%%%
%%%%%
\end{document}
