\documentclass[12pt]{article}
\usepackage{pmmeta}
\pmcanonicalname{MultivariateGammaFunctionrealvalued}
\pmcreated{2013-03-22 14:22:06}
\pmmodified{2013-03-22 14:22:06}
\pmowner{rspuzio}{6075}
\pmmodifier{rspuzio}{6075}
\pmtitle{multivariate gamma function (real-valued)}
\pmrecord{15}{35853}
\pmprivacy{1}
\pmauthor{rspuzio}{6075}
\pmtype{Definition}
\pmcomment{trigger rebuild}
\pmclassification{msc}{62H10}
%\pmkeywords{Gamma multivariate real}
\pmdefines{gamma function (multivariate real)}

% this is the default PlanetMath preamble.  as your knowledge
% of TeX increases, you will probably want to edit this, but
% it should be fine as is for beginners.

% almost certainly you want these
\usepackage{amssymb}
\usepackage{amsmath}
\usepackage{amsfonts}

% used for TeXing text within eps files
%\usepackage{psfrag}
% need this for including graphics (\includegraphics)
%\usepackage{graphicx}
% for neatly defining theorems and propositions
%\usepackage{amsthm}
% making logically defined graphics
%%%\usepackage{xypic}

% there are many more packages, add them here as you need them

% define commands here
\DeclareMathOperator{\Tr}{Tr}
\begin{document}
The real-valued multivariate gamma function is defined by
\begin{equation}
\Gamma_m(a) = \int_{\mathfrak{S}} e^{-\Tr S} \left|S\right|^{a-{1 \over 2}(m+1)}\, {\rm d} S,
\end{equation}

where $\mathfrak{S}$ is the set of all $m \times m$ real, positive definite symmetric matrices, i.e.
\begin{equation}
\mathfrak{S} = \left\{S \in \Bbb{R}^{m \times m} \mid S > 0, x^{\rm T}Sx > 0\, \forall\, x \in \mathbb{R}^{m \times 1}\setminus\{ 0\}\right\}.
\end{equation}
The real-valued multivariate gamma function can also be expressed in terms of the gamma function as follows

\begin{equation}
\Gamma_m(a) = \pi^{{1 \over 4} m (m-1)} \prod\limits_{i=1}^{m}\Gamma\left(a-{1 \over 2}(i-1)\right).
\end{equation}

\subsection*{Reference}
A. T. James, ``Distributions of matrix variates and latent roots derived from normal samples,'' {\it Ann. Math. Statist.}, vol. 35, pp. 475-501, 1964.
%%%%%
%%%%%
\end{document}
