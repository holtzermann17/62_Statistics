\documentclass[12pt]{article}
\usepackage{pmmeta}
\pmcanonicalname{Percentile}
\pmcreated{2013-03-22 16:17:13}
\pmmodified{2013-03-22 16:17:13}
\pmowner{CWoo}{3771}
\pmmodifier{CWoo}{3771}
\pmtitle{percentile}
\pmrecord{17}{38402}
\pmprivacy{1}
\pmauthor{CWoo}{3771}
\pmtype{Definition}
\pmcomment{trigger rebuild}
\pmclassification{msc}{62-07}
\pmsynonym{first quartile}{Percentile}
\pmsynonym{third quartile}{Percentile}
\pmsynonym{IQR}{Percentile}
\pmdefines{quartile}
\pmdefines{upper quartile}
\pmdefines{lower quartile}
\pmdefines{interquartile range}

\usepackage{amssymb,amscd}
\usepackage{amsmath}
\usepackage{amsfonts}

% used for TeXing text within eps files
%\usepackage{psfrag}
% need this for including graphics (\includegraphics)
%\usepackage{graphicx}
% for neatly defining theorems and propositions
%\usepackage{amsthm}
% making logically defined graphics
%%\usepackage{xypic}
\usepackage{pst-plot}
\usepackage{psfrag}

% define commands here

\begin{document}
Given a distribution  function $F_X$ of a random variable $X$, on a probability space $(\Omega, B, P)$ a \emph{$p^{\text{th}}$-percentile} of $F_X$ for a given real number $p$, is a real number $r$ such that 
\begin{enumerate}
\item
$\displaystyle P(X\leq r)\geq \frac{p}{100},$
\item
$\displaystyle P(X\geq r)\geq 1-\frac{p}{100}.$
\end{enumerate}


\textbf{Remarks}.  
\begin{itemize}
\item
The most common percentiles of a distribution function are the \PMlinkname{median}{MedianOfADistribution} (the $50^{\text{th}}$-percentile or the second quartile), the \emph{lower quartile} (the $25^{\text{th}}$-percentile or the first quartile), and the \emph{upper quartile} (the $75^{\text{th}}$-percentile or the third quartile).  
\item
In practice, the quartiles are calculated as follows: calculate the median $m$ first, then the median of the data points below $m$ is the first quartile, and the median of the data points above $m$ is the third quartile.
\item
The interval between the first quartile and the third quartile is called the \emph{interquartile range}, or \emph{IQR} for short.  Sometimes, the difference between the first and third quartiles is also called the IQR.  Like standard deviation, IQR is a measure of dispersion.  However, IQR is a more robust statistic.
\end{itemize}
%%%%%
%%%%%
\end{document}
