\documentclass[12pt]{article}
\usepackage{pmmeta}
\pmcanonicalname{ProofOfExpectedValueOfTheHypergeometricDistribution}
\pmcreated{2013-03-22 13:27:44}
\pmmodified{2013-03-22 13:27:44}
\pmowner{mathwizard}{128}
\pmmodifier{mathwizard}{128}
\pmtitle{proof of expected value of the hypergeometric distribution}
\pmrecord{8}{34029}
\pmprivacy{1}
\pmauthor{mathwizard}{128}
\pmtype{Proof}
\pmcomment{trigger rebuild}
\pmclassification{msc}{62E15}

\endmetadata

% this is the default PlanetMath preamble.  as your knowledge
% of TeX increases, you will probably want to edit this, but
% it should be fine as is for beginners.

% almost certainly you want these
\usepackage{amssymb}
\usepackage{amsmath}
\usepackage{amsfonts}

% used for TeXing text within eps files
%\usepackage{psfrag}
% need this for including graphics (\includegraphics)
%\usepackage{graphicx}
% for neatly defining theorems and propositions
%\usepackage{amsthm}
% making logically defined graphics
%%%\usepackage{xypic}

% there are many more packages, add them here as you need them

% define commands here
\begin{document}
We will first prove a useful property of binomial coefficients. We know
$${n\choose k}=\frac{n!}{k!(n-k)!}.$$
This can be transformed to
\begin{equation}\label{eq:binomial}
{n\choose k}=\frac{n}{k}\frac{(n-1)!}{(k-1)!(n-1-(k-1))!}=\frac{n}{k}{n-1\choose k-1}.
\end{equation}
Now we can start with the definition of the expected value:
$$E[X]=\sum_{x=0}^n \frac{x{K\choose x}{M-K\choose n-x}}{{M\choose n}}.$$
Since for $x=0$ we add a $0$ in this \PMlinkescapetext{formula} we can say
$$E[X]=\sum_{x=1}^n \frac{x{K\choose x}{M-K\choose n-x}}{{M\choose n}}.$$
Applying equation (\ref{eq:binomial}) we get:
$$E[X]=\frac{nK}{M}\sum_{x=1}^n \frac{{K-1\choose x-1}{M-1-(K-1)\choose n-1-(x-1)}}{{M-1\choose n-1}}.$$
Setting $l:=x-1$ we get:
$$E[X]=\frac{nK}{M}\sum_{l=0}^{n-1}\frac{{K-1\choose l}{M-1-(K-1)\choose n-1-l}}{{M-1\choose n-1}}.$$
The sum in this equation is $1$ as it is the sum over all probabilities of a hypergeometric distribution. Therefore we have
$$E[X]=\frac{nK}{M}.$$
%%%%%
%%%%%
\end{document}
