\documentclass[12pt]{article}
\usepackage{pmmeta}
\pmcanonicalname{LehmannScheffeTheorem}
\pmcreated{2013-03-22 16:31:59}
\pmmodified{2013-03-22 16:31:59}
\pmowner{CWoo}{3771}
\pmmodifier{CWoo}{3771}
\pmtitle{Lehmann-Scheff\'e theorem}
\pmrecord{13}{38714}
\pmprivacy{1}
\pmauthor{CWoo}{3771}
\pmtype{Theorem}
\pmcomment{trigger rebuild}
\pmclassification{msc}{62F10}
\pmsynonym{Lehmann-Scheffe theorem}{LehmannScheffeTheorem}
\pmdefines{complete statistic}

\usepackage{amssymb,amscd}
\usepackage{amsmath}
\usepackage{amsfonts}
\usepackage{mathrsfs}

% used for TeXing text within eps files
%\usepackage{psfrag}
% need this for including graphics (\includegraphics)
%\usepackage{graphicx}
% for neatly defining theorems and propositions
\usepackage{amsthm}
% making logically defined graphics
%%\usepackage{xypic}
\usepackage{pst-plot}
\usepackage{psfrag}

% define commands here
\newtheorem{prop}{Proposition}
\newtheorem{thm}{Theorem}
\newtheorem{ex}{Example}
\newcommand{\real}{\mathbb{R}}
\newcommand{\pdiff}[2]{\frac{\partial #1}{\partial #2}}
\newcommand{\mpdiff}[3]{\frac{\partial^#1 #2}{\partial #3^#1}}

\begin{document}
\PMlinkescapeword{complete}

A statistic $S(\boldsymbol{X})$ on a random sample of data $\boldsymbol{X}=(X_1,\ldots,X_n)$ is said to be a \emph{complete statistic} if for any Borel measurable function $g$, $$E(g(S))=0\quad\mbox{implies}\quad P(g(S)=0)=1.$$
In other words, $g(S)=0$ almost everywhere whenever the expected value of $g(S)$ is $0$.  If $S(\boldsymbol{X})$ is associated with a family $f(x\mid \theta)$ of probability density functions (or mass function in the discrete case), then completeness of $S$ means that $g(S)=0$ almost everywhere whenever $E_{\theta}(g(S))=0$ for every $\theta$.

\begin{thm} [Lehmann-Scheff\'e]  If $S(\boldsymbol{X})$ is a complete sufficient statistic and $h(\boldsymbol{X})$ is an unbiased estimator for $\theta$, then, given $$h_0(s) = E(h(\boldsymbol{X}) | S(\boldsymbol{X})=s),$$ $h_0(S)=h_0(S(\boldsymbol{X}))$ is a uniformly minimum variance unbiased estimator of $\theta$.  Furthermore, $h_0(S)$ is unique almost everywhere for every $\theta$.  \end{thm}

%%%%%
%%%%%
\end{document}
