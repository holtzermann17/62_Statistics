\documentclass[12pt]{article}
\usepackage{pmmeta}
\pmcanonicalname{ExponentialRandomVariable}
\pmcreated{2013-03-22 11:54:23}
\pmmodified{2013-03-22 11:54:23}
\pmowner{mathcam}{2727}
\pmmodifier{mathcam}{2727}
\pmtitle{exponential random variable}
\pmrecord{9}{30528}
\pmprivacy{1}
\pmauthor{mathcam}{2727}
\pmtype{Definition}
\pmcomment{trigger rebuild}
\pmclassification{msc}{62E15}
\pmclassification{msc}{06F20}
\pmclassification{msc}{11B65}
\pmclassification{msc}{05C15}
\pmsynonym{exponential distribution}{ExponentialRandomVariable}

\endmetadata

\usepackage{amssymb}
\usepackage{amsmath}
\usepackage{amsfonts}
\usepackage{graphicx}
%%%%\usepackage{xypic}
\begin{document}
$X$ is a \emph{exponential random variable} with parameter $\lambda>0$ if its probability density function is given for $x>0$ by

\begin{align*}
f_X(x) = \lambda e^{-\lambda x}.
\end{align*}

To denote this, one usually writes $X\sim Exp(\lambda)$.


For an exponential random variable $X$:
\begin{enumerate}
\item $X$ is commonly used to model lifetimes and duration between Poisson events.
\item The expected value of $X$ is given by $E[X] = \frac{1}{\lambda}$
\item The variance of $X$ is given by $Var[X] = \frac{1}{\lambda^2}$
\item The moments of $X$ are given by $M_X(t) = \frac{\lambda}{\lambda - t}$
\item It is interesting to note that $X$ is a gamma random variable with an $\alpha$ parameter of 1.

\end{enumerate}
%%%%%
%%%%%
%%%%%
%%%%%
\end{document}
