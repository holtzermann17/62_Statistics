\documentclass[12pt]{article}
\usepackage{pmmeta}
\pmcanonicalname{Estimator}
\pmcreated{2013-03-22 14:52:22}
\pmmodified{2013-03-22 14:52:22}
\pmowner{CWoo}{3771}
\pmmodifier{CWoo}{3771}
\pmtitle{estimator}
\pmrecord{6}{36549}
\pmprivacy{1}
\pmauthor{CWoo}{3771}
\pmtype{Definition}
\pmcomment{trigger rebuild}
\pmclassification{msc}{62A01}
\pmdefines{estimate}

% this is the default PlanetMath preamble.  as your knowledge
% of TeX increases, you will probably want to edit this, but
% it should be fine as is for beginners.

% almost certainly you want these
\usepackage{amssymb,amscd}
\usepackage{amsmath}
\usepackage{amsfonts}

% used for TeXing text within eps files
%\usepackage{psfrag}
% need this for including graphics (\includegraphics)
%\usepackage{graphicx}
% for neatly defining theorems and propositions
%\usepackage{amsthm}
% making logically defined graphics
%%%\usepackage{xypic}

% there are many more packages, add them here as you need them

% define commands here
\begin{document}
\PMlinkescapeword{strictly}

Let $X_1,X_2,\ldots,X_n$ be samples (with observations $X_i=x_i$) from a distribution with probability density function $f(X\mid\theta)$, where $\theta$ is a real-valued unknown \PMlinkname{parameter}{StatisticalModel} in $f$.  Consider $\theta$ as a random variable and let $\tau(\theta)$ be its realization.  

An \emph{estimator} for $\theta$ is a statistic $\eta_{\theta}=\eta_{\theta}(X_1,X_2,\ldots,X_n)$ that is used to, loosely speaking, estimate $\tau(\theta)$.  Any value $\eta_{\theta}(X_1=x_1,X_2=x_2,\ldots,X_n=x_n)$ of $\eta_{\theta}$ is called an \emph{estimate} of $\tau(\theta)$.

\textbf{Example}.
Let $X_1,X_2,\ldots,X_n$ be iid from a normal distribution $N(\mu,\sigma^2)$.  Here the two parameters are the mean $\mu$ and the variance $\sigma^2$.  The sample mean $\overline{X}$ is an estimator of $\mu$, while the sample variance $s^2$ is an estimator of $\sigma^2$.  In addition, sample median, sample mode, sample trimmed mean are all estimators of $\mu$.  The statistic defined by 
$$\frac{1}{n-1}\sum_{i=1}^{n}(X_i-m)^2,$$
where $m$ is a sample median, is another estimator of $\sigma^2$.
%%%%%
%%%%%
\end{document}
