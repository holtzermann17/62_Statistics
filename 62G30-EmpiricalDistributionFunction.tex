\documentclass[12pt]{article}
\usepackage{pmmeta}
\pmcanonicalname{EmpiricalDistributionFunction}
\pmcreated{2013-03-22 14:33:27}
\pmmodified{2013-03-22 14:33:27}
\pmowner{CWoo}{3771}
\pmmodifier{CWoo}{3771}
\pmtitle{empirical distribution function}
\pmrecord{7}{36110}
\pmprivacy{1}
\pmauthor{CWoo}{3771}
\pmtype{Definition}
\pmcomment{trigger rebuild}
\pmclassification{msc}{62G30}

\endmetadata

% this is the default PlanetMath preamble.  as your knowledge
% of TeX increases, you will probably want to edit this, but
% it should be fine as is for beginners.

% almost certainly you want these
\usepackage{amssymb,amscd}
\usepackage{amsmath}
\usepackage{amsfonts}

% used for TeXing text within eps files
%\usepackage{psfrag}
% need this for including graphics (\includegraphics)
%\usepackage{graphicx}
% for neatly defining theorems and propositions
%\usepackage{amsthm}
% making logically defined graphics
%%%\usepackage{xypic}

% there are many more packages, add them here as you need them

% define commands here
\begin{document}
Let $X_1,\ldots,X_n$ be random variables with realizations $x_i=X_i(\omega)\in\mathbb{R}$, $i=1,\ldots,n$.  The \emph{empirical distribution function} $F_n(x,\omega)$ based on $x_1,\ldots,x_n$ is
$$F_n(x,\omega)=\frac{1}{n}\sum_{i=1}^{n}\chi_{A_i}(x,\omega),$$
where $\chi_{A_i}$ is the indicator function (or characteristic function) and $A_i=\lbrace(x,\omega)\mid x_i\leq x \rbrace$.  Note that each indicator function is itself a random variable.
\par
The empirical function can be alternatively and equivalently defined by using the order statistics $X_{(i)}$ of $X_i$ as:
$$
F_n(x,\omega)= 
\begin{cases}
0 & \text{if $x<x_{(1)}$;}\\
\frac{1}{n} & \text{if $x_{(1)}\leq x<x_{(2)}$, $1\leq k<2$;}\\
\frac{2}{n} & \text{if $x_{(2)}\leq x<x_{(3)}$, $2\leq k<3$;}\\
\vdots\\
\frac{i}{n} & \text{if $x_{(i)}\leq x<x_{(i+1)}$, $i\leq k<i+1$;}\\
\vdots\\
1 & \text{if $x\geq x_{(n)}$;}
\end{cases}
$$
where $x_{(i)}$ is the realization of the random variable $X_{(i)}$ with outcome $\omega$.
%%%%%
%%%%%
\end{document}
