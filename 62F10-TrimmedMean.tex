\documentclass[12pt]{article}
\usepackage{pmmeta}
\pmcanonicalname{TrimmedMean}
\pmcreated{2013-03-22 14:42:02}
\pmmodified{2013-03-22 14:42:02}
\pmowner{CWoo}{3771}
\pmmodifier{CWoo}{3771}
\pmtitle{trimmed mean}
\pmrecord{5}{36312}
\pmprivacy{1}
\pmauthor{CWoo}{3771}
\pmtype{Definition}
\pmcomment{trigger rebuild}
\pmclassification{msc}{62F10}
\pmclassification{msc}{62F35}
\pmdefines{winsorized mean}
\pmdefines{outlier}
\pmdefines{robust estimation}

\endmetadata

% this is the default PlanetMath preamble.  as your knowledge
% of TeX increases, you will probably want to edit this, but
% it should be fine as is for beginners.

% almost certainly you want these
\usepackage{amssymb,amscd}
\usepackage{amsmath}
\usepackage{amsfonts}

% used for TeXing text within eps files
%\usepackage{psfrag}
% need this for including graphics (\includegraphics)
%\usepackage{graphicx}
% for neatly defining theorems and propositions
%\usepackage{amsthm}
% making logically defined graphics
%%%\usepackage{xypic}

% there are many more packages, add them here as you need them

% define commands here
\begin{document}
Let $x_1,x_2,\ldots,x_n$ be a set of real-valued data observations.  Let $x_{(1)}\leq x_{(2)}\leq \ldots \leq x_{(n)}$ be the order statistics of the observations.  The $k$\emph{th trimmed mean} $\overline{x}_{k}$ is defined as:
$$\overline{x}_{k}=\frac{x_{(k+1)}+x_{(k+2)}+\ldots+x_{(n-k)}}{n-2k}=\frac{1}{n-2k}\sum_{i=k+1}^{n-k}x_{(i)}.$$
By ordering the original observations, and taking away the first $k$ smallest observations and the first $k$ largest obserations, the trimmed mean takes the arithmetic average of the resulting data.  The idea of a trimmed mean is to eliminate \emph{outliers}, or extreme observations that do not seem to have any logical explanations in calculating the overall mean of a population.
\par
For example, suppose 10 new lightbulbs are drawn from a population of 100 to find the average lifetime of a typical lightbulb, measured in number of hours.  The measurements are 802, 854, 823, 428, 815, 840, 833, 809, 843, 821.  The (arithmetic) mean of the measurement is
$$\frac{802+854+823+428+815+840+833+809+843+821}{10}=786.8,$$
with sample standard deviation = 127.1, whereas the 1st trimmed mean gives:
$$\frac{802+823+815+840+833+809+843+821}{8}=823.25,$$
with sample standard deviation = 14.6, greatly reducing the sample deviation.  
\par
The trimmed mean gives a much more robust estimation (an estimation not greatly affected by outliers) of the average than the arithmetic mean.
\par
Another robust estimtor of a mean is the \emph{winsorized mean}.  Like the trimmed mean, the winsorized mean eliminates the outliers at both ends of an ordered set of observations.  Unlike the trimmed mean, the winsorized mean replaces the outliers with observed values, rather than discarding them.  The formal definition of the $k$th winsorized mean $w_k$ is:
$$w_k=\frac{(k+1)x_{(k+1)}+x_{(k+2)}+\ldots+x_{(n-k-1)}+(k+1)x_{(n-k)}}{n}=\frac{kx_{(k+1)}+(n-2k)\overline{x}_{k}+kx_{(n-k)}}{n}.$$
From the definition, we see that the winsorized mean is the average of the observations where the first $k$ smallest values are replaced by the $k+1$th smallest  value, $x_{(k+1)}$, and the first $k$ largest values are replaced by the $k+1$th largest value, $x_{(n-k)}$.
\par
From the above example, the 1st winsorized mean is
$$\frac{802+843+823+802+815+840+833+809+843+821}{10}=823.1,$$
with sample standard deviation = 16.1, fairly close to the answer given by the trimmed mean.
%%%%%
%%%%%
\end{document}
