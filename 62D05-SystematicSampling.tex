\documentclass[12pt]{article}
\usepackage{pmmeta}
\pmcanonicalname{SystematicSampling}
\pmcreated{2013-03-22 15:15:23}
\pmmodified{2013-03-22 15:15:23}
\pmowner{CWoo}{3771}
\pmmodifier{CWoo}{3771}
\pmtitle{systematic sampling}
\pmrecord{4}{37039}
\pmprivacy{1}
\pmauthor{CWoo}{3771}
\pmtype{Definition}
\pmcomment{trigger rebuild}
\pmclassification{msc}{62D05}

\usepackage{amssymb,amscd}
\usepackage{amsmath}
\usepackage{amsfonts}

% used for TeXing text within eps files
%\usepackage{psfrag}
% need this for including graphics (\includegraphics)
%\usepackage{graphicx}
% for neatly defining theorems and propositions
%\usepackage{amsthm}
% making logically defined graphics
%%%\usepackage{xypic}

% define commands here
\begin{document}
Systematic sampling is a method of sampling $n$ items from a
population of $N$ items with the following sequence of procedures:
\begin{enumerate}
\item label each item in the population as $x_i$, where $1\leq i\leq
N$,
\item randomly select one item from the first $k$ items in the
population, where $k=\lceil N/n \rceil$, the smallest integer
greater than or equal to $N/n$,
\item pick the $k$th item thereafter, $x_i,x_{i+k},x_{i+2k},\ldots$,
until all $n$ items are picked.
\end{enumerate}
If $N\equiv0 \mod n$, then all $n$ items can be picked before
reaching the end of the population.  Otherwise, define
$x_{j+N}:=x_j$.  Using this, we can continue to pick our sample
units until all $n$ items are picked.
\\\\
The above method suggests that we do not have to make our first pick
from among the first $k$ items.  By the above definition, we can
start anywhere in the population and still end up $n$ units of
sample.
\\\\
\textbf{Remarks.}
\begin{itemize}
\item If $n$ divides $N$, then systematic sampling can be viewed as
grouping the population into $k=N/n$ strata, and picking one sample
from each stratum.  The difference between systematic sampling and
stratified sampling is that in systematic sampling, only the first
sample is picked randomly, all other samples are picked based on the
position of the first pick.
\item Again, if $n\mid N$, one can view systematic sampling as a
one-stage cluster sampling, where a primary sampling unit is defined
as the set of units $x_i,x_{i+k},\ldots,x_{i+(n-1)k}$ in the
population, where $k=N/n$ and $1\leq i\leq k$.  In this way, there
are $k$ primary sampling units.  A simple random sample of one unit
can then be drawn from these $k$ units.  All of the items within the
selected primary sample will then be the complete sample drawn by
systematic sampling.
\item If some linear trend exists in the population, it is a good
idea to order the population so the linearity is preserved by the
order, so that when a systematic sampling is performed, the
linearity gets carried over to the samples.
\item When ordering the population, care should be taken so that there
does not exhibit any periodic patterns (seasonality, etc...) that
would greatly bias the sample.
\end{itemize}
%%%%%
%%%%%
\end{document}
