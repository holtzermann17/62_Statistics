\documentclass[12pt]{article}
\usepackage{pmmeta}
\pmcanonicalname{CramersV}
\pmcreated{2013-03-22 15:10:11}
\pmmodified{2013-03-22 15:10:11}
\pmowner{CWoo}{3771}
\pmmodifier{CWoo}{3771}
\pmtitle{Cramer's V}
\pmrecord{7}{36920}
\pmprivacy{1}
\pmauthor{CWoo}{3771}
\pmtype{Definition}
\pmcomment{trigger rebuild}
\pmclassification{msc}{62H17}
\pmsynonym{Cram\'{e}r's V}{CramersV}
\pmsynonym{phi coefficient}{CramersV}
\pmdefines{phi statistic}

\endmetadata

% this is the default PlanetMath preamble.  as your knowledge
% of TeX increases, you will probably want to edit this, but
% it should be fine as is for beginners.

% almost certainly you want these
\usepackage{amssymb,amscd}
\usepackage{amsmath}
\usepackage{amsfonts}
\usepackage{tabls}
% used for TeXing text within eps files
%\usepackage{psfrag}
% need this for including graphics (\includegraphics)
%\usepackage{graphicx}
% for neatly defining theorems and propositions
%\usepackage{amsthm}
% making logically defined graphics
%%%\usepackage{xypic}

% there are many more packages, add them here as you need them

% define commands here
\begin{document}
\PMlinkescapeword{cell}
\PMlinkescapeword{theory}
\PMlinkescapeword{represents}
\PMlinkescapeword{calculate}
\PMlinkescapeword{order}
\PMlinkescapeword{categories}
\PMlinkescapeword{category}
\PMlinkescapeword{classes}
\PMlinkescapeword{information}

Cramer's V is a statistic measuring the strength of association or
dependency between two (nominal) categorical variables in a
contingency table.
\\\\
\textbf{Setup.} Suppose $X$ and $Y$ are two categorical variables
that are to be analyzed in a some experimental or observational data
with the following information:
\begin{itemize}
\item $X$ has $M$ distinct categories or classes, labeled
$X_1,\ldots,X_M$,
\item $Y$ has $N$ distinct categories, labeled $Y_1,\ldots,Y_N$,
\item $n$ pairs of observations $(x_k,y_k)$ are taken, where $x_i$
belongs to one of the $M$ categories in $X$ and $y_i$ belongs to one
of the $N$ categories in $Y$.
\end{itemize}
Form a $M\times N$ contingency table such that Cell $(i,j)$ contains
the count $n_{ij}$ of occurrences of Category $X_i$ in $X$ and
Category $Y_j$ in $Y$:
\begin{center}
\begin{tabular}{|c|c|c|c|c|}
\hline
$X\backslash Y$ & $Y_1$ & $Y_2$ & $\cdots$ & $Y_N$ \\
\hline
$X_1$ & $n_{11}$ & $n_{12}$ & $\cdots$ & $n_{1N}$ \\
\hline
$X_2$ & $n_{21}$ & $n_{22}$ & $\cdots$ & $n_{2N}$ \\
\hline
$\vdots$ & $\vdots$ & $\vdots$ & $\ddots$ & $\vdots$ \\
\hline
$X_M$ & $n_{M1}$ & $n_{M2}$ & $\cdots$ & $n_{MN}$ \\
\hline
\end{tabular}
\end{center}
Note that $n=\sum n_{ij}$.
\\\\
\textbf{Definition.} Suppose that the null hypothesis is that $X$
and $Y$ are independent random variables.  Based on the table and
the null hypothesis, the chi-squared statistic $\chi^2$ can be
computed. Then, \emph{Cramer's V} is defined to be
$$V=V(X,Y)=\sqrt{\frac{\chi^2}{n\operatorname{min}(M-1,N-1)}}.$$  Of course,
in order for $V$ to make sense, each categorical variable must have
at least 2 categories.
\\\\
\textbf{Remarks.}
\begin{enumerate}
\item $0\leq V\leq 1$.  The closer $V$ is to 0, the smaller the
association between the categorical variables $X$ and $Y$.  On the
other hand, $V$ being close to 1 is an indication of a strong
association between $X$ and $Y$.  If $X=Y$, then $V(X,Y)=1$.
\item When comparing more than two categorical variables, it is
customary to set up a square matrix, where cell $(i,j)$ represents
the Cramer's V between the $i$th variable and the $j$th variable. If
there are $n$ variables, there are $\frac{n(n-1)}{2}$ Cramer's V's
to calculate, since, for any discrete random variables $X$ and $Y$,
$V(X,X)=1$ and $V(X,Y)=V(Y,X)$. Consequently, this matrix is
symmetric.
\item If one of the categorical variables is dichotomous, (either $M$ or $N=2$), Cramer's V is equal to the \emph{phi statistic} ($\Phi$), which is defined to be $$\Phi=\sqrt{\frac{\chi^2}{n}}.$$
\item Cramer's V is named after the Swedish
mathematician and statistician Harald Cram\'{e}r, who sought to make
statistics mathematically rigorous, much like Kolmogorov's
axiomatization of probability theory.  Cram\'{e}r also made
contributions to number theory, probability theory, and actuarial
mathematics widely used by the insurance industry.
\end{enumerate}
\begin{thebibliography}{8}
\bibitem{agresti} A. Agresti, \emph{Categorical Data Analysis}, Wiley-Interscience, 2nd ed. 2002.
\bibitem{cramer} H. Cram\'{e}r, \emph{Mathematical Methods of Statistics}, Princeton University Press, 1999.
\end{thebibliography}
%%%%%
%%%%%
\end{document}
