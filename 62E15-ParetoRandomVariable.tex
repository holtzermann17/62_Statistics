\documentclass[12pt]{article}
\usepackage{pmmeta}
\pmcanonicalname{ParetoRandomVariable}
\pmcreated{2013-03-22 12:34:05}
\pmmodified{2013-03-22 12:34:05}
\pmowner{alozano}{2414}
\pmmodifier{alozano}{2414}
\pmtitle{Pareto random variable}
\pmrecord{9}{32816}
\pmprivacy{1}
\pmauthor{alozano}{2414}
\pmtype{Definition}
\pmcomment{trigger rebuild}
\pmclassification{msc}{62E15}
\pmsynonym{Pareto distribution}{ParetoRandomVariable}

\endmetadata

% this is the default PlanetMath preamble.  as your knowledge
% of TeX increases, you will probably want to edit this, but
% it should be fine as is for beginners.

% almost certainly you want these
\usepackage{amssymb}
\usepackage{amsmath}
\usepackage{amsfonts}

% used for TeXing text within eps files
%\usepackage{psfrag}
% need this for including graphics (\includegraphics)
%\usepackage{graphicx}
% for neatly defining theorems and propositions
%\usepackage{amsthm}
% making logically defined graphics
%%%\usepackage{xypic} 

% there are many more packages, add them here as you need them

% define commands here
\begin{document}
 $X$ is a \textbf{Pareto random variable} with parameters \textbf{$a, b$} if

$f_X(x) = \frac{a b^a}{x^{a+1}} $, $x\in[b,\infty)$

Parameters:

\begin{list}{$\star$ }{}
\item $a \in (0,\infty)$
\item $b \in (0,\infty)$
\end{list}

Syntax:

$X\sim Pareto(a,b)$

Notes:

\begin{enumerate}
\item $X$ represents a random variable with shape parameter $a$ and scale parameter $b$.
\item The expected value of $X$ is noted as $E[X] = \frac{a b}{a-1}$ with $a\in\{2,3,\ldots\}$
\item The variance of $X$ is noted as $Var[X] = \frac{a b^2}{(a-1)^2 (a-2)}$, $a \in \{3,4,...\}$
\item The cumulative distribution function of $X$ is noted as $F(x) = 1 - (\frac{b}{x})^a$
\item The moments of $X$ around 0 are noted as $E[X^n] = \frac{a b^n}{a-n}$, $n \in \{1,2,...,a-1\}$
\end{enumerate}
%%%%%
%%%%%
\end{document}
