\documentclass[12pt]{article}
\usepackage{pmmeta}
\pmcanonicalname{ProbabilityProblem}
\pmcreated{2013-03-22 19:11:21}
\pmmodified{2013-03-22 19:11:21}
\pmowner{statsCab}{25915}
\pmmodifier{statsCab}{25915}
\pmtitle{probability problem}
\pmrecord{4}{42100}
\pmprivacy{1}
\pmauthor{statsCab}{25915}
\pmtype{Definition}
\pmcomment{trigger rebuild}
\pmclassification{msc}{62-01}

% this is the default PlanetMath preamble.  as your knowledge
% of TeX increases, you will probably want to edit this, but
% it should be fine as is for beginners.

% almost certainly you want these
\usepackage{amssymb}
\usepackage{amsmath}
\usepackage{amsfonts}

% used for TeXing text within eps files
%\usepackage{psfrag}
% need this for including graphics (\includegraphics)
%\usepackage{graphicx}
% for neatly defining theorems and propositions
%\usepackage{amsthm}
% making logically defined graphics
%%%\usepackage{xypic}

% there are many more packages, add them here as you need them

% define commands here

\begin{document}
This is in response to the following request:

A parent particle divides into 0,1,or 2 particles with probabilities 1/4,1/2,1/4.it disappears after splitting.let Xn denotes the number of particles in n-th generations with X0=1.find P(X2>0) and the probabilities that X1=2 given that X2=1. 

http://planetmath.org/?op=getobj;from=requests;id=927

For my first entry I will try to answer the question.

Let $p_0, p_1$ and $p_2$ be the nonzero probabilities of dividing into 0, 1, or 2 particles, and let $X_n$ denotes the number of particles at the $n^{th}$ generation.

With $X_0 =1$, find 1) $P(X_2 > 2)$ and 2) $P(X_1 = 2 | X_2 = 1)$

1) After two generations there can be at most $2^2$ particles so $P(X_2 >2) = P(X_2 = 3) + P(X_2 =4)$
$$P(X_2 = 4) = p_2^2$$
$$P(X_2 = 3) = 2 p_1 p_2^2$$
Note that if $X_2 = 3$, then $X_2 = 2$.

$$P(X_2 > 2) = p_2^2(1 + 2 p_1)$$

Using your values I get 3/32.

2) From the definition of conditional probability 
$$P(X_1|X_2) = \frac{P(X_1 \cap X_2)}{P(X_2)}$$ 

First 
$$P(X_2 = 1) = p_1^2  +  2 p_0 p_1 p_2$$

Why?
To get to $X_2 = 1$, at $n=1$ there are either one or two particles, if there is one particle it remains one at $n=2$, and if there were two particles at $n=1$, then one has to go to zero and the other one---this can happen two ways.

Finally $P(X_1 =1 \cap X_2) = p_1 p_2$.

$$P(X_1 = 2 | X_2 = 1) = \frac{p_2}{p_1 + 2 p_0 p_2}$$

Using your values I get 2/3.

Now I have a question for you to think about. What happens in the long run, as $n \rightarrow \infty$?
%%%%%
%%%%%
\end{document}
